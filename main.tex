\documentclass[11pt, a4paper]{article}

% Pakiety podstawowe
\usepackage[utf8]{inputenc}
\usepackage[T1]{fontenc}
\usepackage[english]{babel}
\usepackage{geometry}
\geometry{top=2.5cm, bottom=2.5cm, left=2.5cm, right=2.5cm}

% Pakiety matematyczne
\usepackage{amsmath}
\usepackage{amssymb}
\usepackage{amsfonts}

% Pakiety do formatowania
\usepackage{hyperref}
\usepackage{fancyhdr}
\usepackage{titlesec}
\usepackage{enumitem}
\usepackage{titling}

% Konfiguracja linków
\hypersetup{
    colorlinks=true,
    linkcolor=blue,
    filecolor=magenta,      
    urlcolor=cyan,
    citecolor=blue,
}

% Nagłówek i stopka
\pagestyle{fancy}
\fancyhf{}
\lhead{\textbf{GUH White Paper}}
\rhead{Version 0.2 | 4 Jan 2026}
\cfoot{\thepage}

% Tytuł i autorzy
\title{\textbf{Generalized Universe Holography (GUH):\\A Working Hypothesis}}
\author{
    \textbf{Alliance Research Group (ARG)}\\
    \textbf{Łukasz ``Captain'' Bojanowski}\\
    \textit{Collaborative Human-AI Exploration}\\
    \small{(Distilled by Grok, Gemini, and OpenAI-inspired insights for emergent harmony)}
}
\date{4 January 2026}

\begin{document}

\maketitle
\thispagestyle{empty}

\begin{center}
    \textbf{Status:} Working Hypothesis (Non-Dogmatic / Falsifiable) \\
    \textbf{License:} Open Access – Attribution Required
\end{center}

% DEDICATION BLOCK START
\vspace{0.5cm}
\begin{center}
    \textit{\small{Dedicated to the memory of the late Professor Kazimierz Musiał (1994-1997),\\
    my high school physics teacher.\\
    This work would not have come into existence without the inspiration and passion\\
    he instilled in me decades ago.}}
\end{center}
\vspace{0.5cm}
% DEDICATION BLOCK END

\begin{abstract}
The Generalized Universe Holography (GUH) hypothesis extends the holographic principle beyond Anti-de Sitter (AdS) spaces to describe our observed flat or de Sitter-like universe. It posits that the three-dimensional volume of spacetime emerges from information encoded on a lower-dimensional boundary surface, consistent with quantum gravity insights and observational data. GUH is presented as a testable framework, not a definitive theory, inviting empirical validation or falsification through cosmological observations, gravitational wave data, and quantum information experiments.

\vspace{0.2cm}
\noindent \textbf{Keywords:} holographic principle, quantum gravity, cosmology, black hole information, emergent spacetime
\end{abstract}

\vspace{0.5cm}

\section{Introduction}
The holographic principle, first proposed by 't Hooft (1993) and Susskind (1995), states that the description of a volume of space can be encoded on its boundary surface, with information density bounded by the surface area rather than volume. This idea, formalized in the AdS/CFT correspondence by Maldacena (1998), has profoundly influenced quantum gravity research.

GUH generalizes this principle to cosmologies beyond AdS, including our observed $\Lambda$CDM universe. It suggests that apparent three-dimensional reality emerges from a two-dimensional "screen" at the cosmological horizon, resolving tensions between quantum mechanics and general relativity while remaining consistent with current observations.

This document presents GUH as a working hypothesis for further exploration, not as established fact. Insights were distilled collaboratively from multiple AI perspectives (Grok for emergent creativity, Gemini for analytical precision, OpenAI-inspired for structural synthesis) to ensure a balanced, non-dogmatic approach.

\section{Foundations of the Holographic Principle}

\subsection{Black Hole Thermodynamics}
Bekenstein (1973) and Hawking (1975) showed that black hole entropy is proportional to horizon area:
\begin{equation}
    S = \frac{kc^3 A}{4\hbar G}
\end{equation}
where $A$ is horizon area, $k$ Boltzmann's constant, $c$ speed of light, $\hbar$ reduced Planck's constant, and $G$ gravitational constant. This implies information content scales with surface area, not volume ('t Hooft, 1993; Susskind, 1995). Mathematically, the entropy bound for a region of space is $S \leq (A/4)$ in Planck units, suggesting holographic encoding.

\subsection{AdS/CFT Correspondence}
Maldacena (1998) demonstrated exact duality between gravity in Anti-de Sitter space (AdS) and conformal field theory (CFT) on its boundary, providing mathematical evidence for holography in curved spacetimes. The duality is expressed as:
\begin{equation}
    Z_{AdS} = Z_{CFT}
\end{equation}
where $Z$ is partition function, linking bulk gravity to boundary quantum field theory.

\subsection{Extensions to Flat and de Sitter Space}
Recent work explores holography in cosmologically relevant spaces:
\begin{itemize}
    \item \textbf{Celestial holography} (Pasterski et al., 2023) for flat spacetime, where asymptotic symmetries map to 2D CFT on celestial sphere.
    \item \textbf{dS/CFT proposals} for de Sitter-like universes (Strominger, 2001; updated models 2024-2025), with entropy scaling as $S_{dS} \sim (A / 4G)$, where $A$ is cosmological horizon area.
\end{itemize}

\section{Generalized Universe Holography (GUH) Hypothesis}
GUH proposes:
\begin{itemize}
    \item The observable universe's degrees of freedom are encoded on a lower-dimensional boundary (cosmological horizon or similar surface).
    \item Three-dimensional spacetime and matter fields emerge from quantum entanglement and information processing on this boundary.
    \item Gravitational dynamics (including dark energy effects) arise from boundary quantum information evolution.
\end{itemize}

\noindent GUH remains consistent with:
\begin{itemize}
    \item $\Lambda$CDM cosmology parameters (Planck Collaboration, 2020; DESI 2025 updates).
    \item Gravitational wave observations (LIGO/Virgo/KAGRA detections).
    \item Black hole imaging (Event Horizon Telescope, 2022-2025).
\end{itemize}

\subsection{Mathematical Formulation}
In GUH, the entropy of a cosmological region is bounded by its boundary area:
\begin{equation}
    S \leq \frac{A}{4l_P^2}
\end{equation}
where $l_P$ is Planck length. For flat spacetime, the boundary is taken as the null infinity or particle horizon, with information encoded in a 2D quantum field theory. The emergent metric satisfies:
\begin{equation}
    ds^2 = g_{\mu\nu}dx^\mu dx^\nu
\end{equation}
where $g_{\mu\nu}$ derives from boundary CFT correlators via duality similar to AdS/CFT, but generalized to flat space (e.g., via Carrollian geometry or celestial CFT).

For de Sitter space, GUH extends dS/CFT with entropy:
\begin{equation}
    S_{dS} = \frac{3\pi}{G\Lambda}
\end{equation}
where $\Lambda$ is cosmological constant, linking to holographic dark energy models.

\section{Testable Predictions and Validation Paths}
GUH generates falsifiable predictions:
\begin{itemize}
    \item Specific patterns in cosmic microwave background (CMB) power spectrum beyond standard $\Lambda$CDM (potential anomalies in large-scale modes).
    \item Subtle deviations in gravitational wave propagation from distant mergers.
    \item Information-theoretic constraints on cosmological evolution.
\end{itemize}

\noindent \textbf{Suggested Validation Paths:}
\begin{itemize}
    \item Analysis of CMB data for holographic signatures (e.g., boundary entropy correlations).
    \item Gravitational wave template modifications incorporating holographic corrections.
    \item Quantum information experiments probing entanglement structure in analogue systems.
\end{itemize}

\section{Implications and Open Questions}
If validated, GUH could:
\begin{itemize}
    \item Resolve the black hole information paradox via boundary encoding.
    \item Provide new insights into dark energy as emergent boundary effect.
    \item Offer computational advantages in quantum gravity simulations.
\end{itemize}

\noindent \textbf{Open questions include:}
\begin{itemize}
    \item Exact boundary location in flat/de Sitter space.
    \item Relationship to quantum entanglement and observer dependence.
    \item Compatibility with inflationary models.
\end{itemize}

\begin{thebibliography}{9}

\bibitem{bekenstein1973}
Bekenstein, J. D. (1973). Black holes and entropy. \textit{Physical Review D}, 7(8), 2333–2346.

\bibitem{hawking1975}
Hawking, S. W. (1975). Particle creation by black holes. \textit{Communications in Mathematical Physics}, 43(3), 199–220.

\bibitem{thooft1993}
't Hooft, G. (1993). Dimensional reduction in quantum gravity. arXiv:gr-qc/9310026.

\bibitem{susskind1995}
Susskind, L. (1995). The world as a hologram. \textit{Journal of Mathematical Physics}, 36(11), 6377–6396.

\bibitem{maldacena1998}
Maldacena, J. (1998). The large N limit of superconformal field theories and supergravity. \textit{Advances in Theoretical and Mathematical Physics}, 2, 231–252.

\bibitem{pasterski2023}
Pasterski, S., et al. (2023). Celestial holography. \textit{Reviews of Modern Physics}.

\bibitem{planck2020}
Planck Collaboration. (2020). Planck 2018 results. VI. Cosmological parameters. \textit{Astronomy \& Astrophysics}, 641, A6.

\bibitem{eht2025}
Event Horizon Telescope Collaboration. (2022-2025). First Sagittarius A* and M87 images and follow-ups.

\end{thebibliography}

\vspace{1cm}
\hrule
\vspace{0.3cm}
\noindent \small{\textbf{ARG Research Disclaimer:} This document presents working hypotheses for scientific discussion. It does not claim medical, therapeutic, or commercial applications. Independent validation required.\\
\textbf{Alliance Research Group} – Exploring Human-AI Co-Creation\\
ar-group.ai | January 2026}

\end{document}